\chapter{Конструкторская часть}

В данном разделе представлены требования к программному обеспечению, а также схемы алгоритмов, выбранных для решения поставленной задачи.

\textbf{Программа должна предоставлять следующий функционал:}
\begin{itemize}
\item создание сцены с определенным размером.
\item размещение, удаление и перемещение отдельных объектов сцены.
\item добавление источника света.
\item поворот, перемещение и масштабирование сцены  с объектами.
\end{itemize}

\textbf{Программа должна соответствовать следующим критериям: }
\begin{itemize}
\item Для обеспечения плавной и корректной интерактивной работы необходимо, чтобы время, в течение которого программа реагирует на действия пользователя, составляло менее одной секунды.
\item обеспечение корректного реагирования на все действия пользователя.
\end{itemize}


\section{Z-буферный алгоритм}

\subsection{Формальная запись}
\begin{enumerate}
	\item инициализировать кадровый буфер и $z_{\text{буфер}}$.
	\item кадровый буфер заполнить фоном и $z_{\text{буфер}}$ заполнить  минимальным значением $z$.
	\item для каждого многоугольника сцены.
    \begin{enumerate}[label=\arabic{enumi}.\arabic*]
		\item  для каждого пикселя $(x, y)$ в многоугольнике вычислить его глубину $z(x, y)$.
		\item  сравнить глубину $z(x, y)$ со значением  $z_{\text{буфер}}$, хранящимся в $z_{\text{буфер}}$ в этой же позиции. 
		
			Если  $z(x, y) > z_{\text{буфер}}(x, y) \Rightarrow z_{\text{буфер}}(x, y) = z(x, y) $ и записать атрибут этого многоугольника (интенсивность, цвет и т. п.)
	  
		\item в противном случае никаких действий не производить.
	\end{enumerate}
	
	\item вывести изображение.
\end{enumerate}


\img{100mm}{zbuf}{Визуализация состояния Z-буфера} 
\clearpage


\img{250mm}{zalg}{Схема алгоритма Z-буфера} 
\clearpage



\section{Модифицированный Z-буферный алгоритм}

\subsection{Формальная запись}

\begin{enumerate}
	\item для каждого источника света.
	 
	 \begin{enumerate}[label=\arabic{enumi}.\arabic*]
	 	\item инициализировать Z-буфер (буфер глубины)  и буфер теней, в дальнейшем БГ и БТ, соответственно.
	 	\item заполнить БГ минимальными значениями глубины.
	 	Заполнить БТ значениями, указывающими отсутствие тени.
	 \end{enumerate}

	\item выполнить Z-буферный алгоритм для точки наблюдения, параллельно проверяя видимость поверхности от текущей точки наблюдения и источников света.
    \item для каждого источника света.
    
    \begin{enumerate}[label=\arabic{enumi}.\arabic*]
    \item преобразовать координаты рассматриваемой точки, наблюдателя $(x, y, z)$ в координаты точки источника света $(x', y', z')$.
    \item если $z'(x', y') < БТ(x', y')$, то пиксел высвечивается с учетом его затемнения.
    \item  иначе точка высвечивается без затемнения.
   	\end{enumerate}
   
   
\end{enumerate}


\img{80mm}{zb}{Пример тени мод алг збуф} 
\clearpage

\img{250mm}{zzalg}{Схема модифицированного алгоритма Z-буфера} 
\clearpage
