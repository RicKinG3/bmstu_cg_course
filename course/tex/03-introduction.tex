\chapter*{ВВЕДЕНИЕ}
\addcontentsline{toc}{chapter}{ВВЕДЕНИЕ}

ПО для макетирования и визуализации загородной местности может применяться в ландшафтном дизайне, архитектуре и планировании территории.
Визуализация при планировании местности улучшает коммуникацию и снижает риск недопонимания между заказчиком и исполнителем~\cite{intro_visualization}.

Цель работы --- разработка программного обеспечения для макетирования и визуализации загородной местности.

Для достижения поставленной цели требуется решить следующие задачи:
\begin{itemize}
	\item формально описать структуру объектов;
	\item выбрать алгоритмы трехмерной графики для визуализации сцены;
	\item спроектировать ПО, позволяющее макетировать и визуализировать загородную местность;
	\item разработать программное обеспечение для макетирования и визуализации загородной местности;
	% todo (Время отрисовки относительно числа объектов) норм 
	\item исследовать затраты реализации по времени выполнения.
\end{itemize}


\if 0
\item выбрать инструменты для реализации разработанного ПО;
	\item разработать выбранные алгоритмы;
\item выбрать структуры данных для объектов сцены;
\item выбрать язык программирования;
\item реализовать выбранные алгоритмы;

\fi