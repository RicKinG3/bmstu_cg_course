\chapter*{Введение}
\addcontentsline{toc}{chapter}{Введение}

Программное моделирование загородной местности с каждым годом набирает все большее значение в области ландшафтного дизайна, архитектуры и планирования. Применение современных технологий дает возможность разработки детализированных трехмерных моделей, обеспечивая тем самым более глубокое понимание и визуализацию географических и ландшафтных особенностей, а также предстоящих изменений. Использование метода 3D-визуализации при планировании местности преобразует сложные концепции в ясные визуальные образы, что значительно упрощает процесс коммуникации и минимизирует риск возникновения недопонимания между заказчиком и исполнителем \cite{intro_visualization}.
 
 
Цель работы --- изучение и выбор алгоритмов для разработки программного обеспечения, предназначенного для  визуализации и макетировании загородной местности. 

%разработка программного обеспечения для визуализации и макетировании загородной местности.

Чтобы достигнуть поставленной цели, требуется решить следующие задачи:
\begin{itemize}
    \item формально описать структуру объектов;
    \item выбрать алгоритмы трехмерной графики для визуализации сцены и объектов;
    \item  формально записать выбранные алгоритмы;
    \item  выбрать структуры данных для объектов сцены.
%    \item Выбрать язык программирования и среду разработки.
%    \item Реализовать выбранные алгоритмы.
%    \item Реализовать программное обеспечения для визуализации и редактирования загородной местности. %
\end{itemize}
