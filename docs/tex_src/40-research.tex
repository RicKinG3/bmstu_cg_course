\chapter{Исследовательская часть}

В данном разделе будут приведены примеры работы разработанного программного обеспечения, а также будет поставлен эксперимент, в котором будут сравнены геометрические характеристики разработанной модели с геометрическими характеристиками бицепса реального человека.

\section{Результаты работы программного обеспечения}

В листинге \ref{lst:conf} описана конфигурация модели мышцы с рисунков \ref{img:ex1} и \ref{img:ex2}, на рисунках \ref{img:ex1} и \ref{img:ex2} представлен пример модели мышцы в полностью растянутом и сокращённом состояниях соответственно, на рисунке \ref{img:control_panel} представлено окно панели управления.

%\begin{lstinputlisting}[
  %      caption={Конфигурация для мышцы из примера.},
   %    label={lst:conf},
   %     style={rust}
    %]{../../config/main.yaml}
%\end{lstinputlisting}
\clearpage
\imgw{140mm}{ex1}{Пример модели мышцы в полностью растянутом состоянии.}
\imgw{140mm}{ex2}{Пример модели мышцы в полностью сокращённом состоянии.}
\clearpage
\img{150mm}{control_panel}{Окно панели управления.}

\section{Постановка эксперимента}

\subsection{Цель эксперимента}

Цель эксперимента - сравнение геометрических характеристик (радиусов узлов) разработанной модели и реального человеческого бицепса в различных положениях мышцы.

\subsection{Данные реального бицепса}

Данные реального бицепса были взяты из проекта OpenArm 2.0 \cite{openarm}. Данный проект предоставляет необработанные и сегментированные ультразвуковые сечения плечевой части руки (в том числе бицепса) \cite{openarm1}. Анализ данных может быть произведен с помощью ПО ITK-SNAP \cite{itk}. На рисунке \ref{img:raw} приведён пример необработанных данных, а на рисунке \ref{img:segmented} - пример сегментированных данных.

\img{150mm}{raw}{Необработанные данные.}
\clearpage
\img{150mm}{segmented}{Сегментированные данные.}

\subsection{Замеры реального бицепса}

На рисунках \ref{img:real1} -- \ref{img:real4} приведены замеры для 9 узлов мышцы в 4 положениях: под углом 0$^\circ$, 30$^\circ$, 60$^\circ$ и 90$^\circ$. Данные замеров сведены в таблицу \ref{tab:real}.

\img{100mm}{real1}{Замеры для мышцы в положении 0$^\circ$.}
\img{100mm}{real2}{Замеры для мышцы в положении 30$^\circ$.}
\img{100mm}{real3}{Замеры для мышцы в положении 60$^\circ$.}
\img{100mm}{real4}{Замеры для мышцы в положении 90$^\circ$.}
\clearpage

\begin{table}[!h]
    \begin{center}
        \begin{tabular}{|c|c|c|c|c|}
            \hline
            Номер узла & 0$^\circ$ & 30$^\circ$ & 60$^\circ$ & 90$^\circ$ \\
            \hline
            \hline
            1 & 7.6 & 16.2 & 7.8 & 9.5 \\
            \hline
            2 & 18.2 & 18.7 & 16.2 & 15.2 \\
            \hline
            3 & 20.7 & 22.5 & 24.9 & 23.4 \\
            \hline
            4 & 21.2 & 26.2 & 32.4 & 27.7 \\
            \hline
            5 & 21.7 & 27.5 & 32.4 & 35.2 \\
            \hline
            6 & 24.3 & 31.3 & 32.0 & 35.8 \\
            \hline
            7 & 25.7 & 28.7 & 29.8 & 36.6 \\
            \hline
            8 & 20.7 & 26.6 & 22.0 & 24.2 \\
            \hline
            9 & 13.2 & 20.8 & 12.7 & 16.7 \\
            \hline
        \end{tabular}
    \end{center}
    \caption{\label{tab:real} Радиусы узлов при различном угле наклона в локте}
\end{table}

\subsection{Замеры разработанной модели}

Для измерения геометрических характеристик модели необходимо сперва ее сконфигурировать. Конфигурация будет выполняться исходя из данных реальной мышцы: начальные значения узлов модели пропорциональны начальным значениям узлов реальной мышцы, коэффициенты роста пропорциональны приросту реальной мышцы при переходе из начального в конечное состояние.

Таким образом конфигурация будет иметь вид, описанный в листинге \ref{lst:real_conf}.

Вид модели представлен на рисунках \ref{img:model1} -- \ref{img:model2}. Результаты представлены в таблице \ref{tab:model}. 

\img{100mm}{model1}{Модель в положении 0$^\circ$.}
\img{100mm}{model2}{Модель в положении 30$^\circ$.}
\img{100mm}{model3}{Модель в положении 60$^\circ$.}
\img{100mm}{model4}{Модель в положении 90$^\circ$.}

\begin{table}[!h]
    \begin{center}
        \begin{tabular}{|c|c|c|c|c|}
            \hline
            Номер узла & 0$^\circ$ & 30$^\circ$ & 60$^\circ$ & 90$^\circ$ \\
            \hline
            \hline
            1 & 7.6 &   7.7 &  7.8 &  8.0 \\
            \hline
            2 & 18.2 & 18.1 & 17.9 & 17.6 \\
            \hline
            3 & 20.7 & 20.8 & 21.0 & 21.2 \\
            \hline
            4 & 21.2 & 21.3 & 21.7 & 22.3 \\
            \hline
            5 & 21.7 & 22.1 & 23.0 & 24.4 \\
            \hline
            6 & 24.3 & 24.6 & 25.4 & 26.6 \\
            \hline
            7 & 25.7 & 26.0 & 26.7 & 27.9 \\
            \hline
            8 & 20.7 & 20.8 & 21.0 & 21.4 \\
            \hline
            9 & 13.2 & 13.3 & 13.5 & 13.9 \\
            \hline
        \end{tabular}
    \end{center}
    \caption{\label{tab:model} Радиусы узлов модели при различном угле наклона в локте}
\end{table}

\subsection{Сравнение замеров}

В таблице \ref{tab:percents} приведены приросты для всех узлов для реального бицепса (РБ) и модели относительно при переходе из состояния $0^\circ$ в состояние $90^\circ$.

\begin{table}[!h]
    \begin{center}
        \begin{tabular}{|c|c|c|c|c|}
            \hline
            Номер узла & РБ & модель \\
            \hline
            \hline
            1 & 1.9 & 0.4 \\
            \hline
            2 & -3.0 & -0.6 \\
            \hline
            3 & 2.7 & 0.5 \\
            \hline
            4 & 6.5 & 1.1 \\
            \hline
            5 & 13.5 & 2.7 \\
            \hline
            6 & 11.5 & 2.3 \\
            \hline
            7 & 10.9 & 2.2 \\
            \hline
            8 & 3.5 & 0.7 \\
            \hline
            9 & 3.5 & 0.7 \\
            \hline
            \hline
        \end{tabular}
    \end{center}
    \caption{\label{tab:percents} Радиусы узлов при различном угле наклона в локте}
\end{table}

Из таблицы видно, что реализованная модель может сохранять задаваемые пропорции, но ввиду того, что модель сохраняет объем, все приращения получились в 5-6 раз меньше.

Из таблиц \ref{tab:real} и \ref{tab:model} можно заметить, что приращения модели при сокращении мышцы строго положительные, в то время как с реальным бицепсом могут быть как положительные, так и отрицательные.


\section*{Вывод}

В данном разделе были рассмотрены примеры работы программного обеспечения, а также были вычислены и сравнены радиусы узлов реального бицепса и сделанной модели. В результате сравнения были получены следующие результаты:
\begin{itemize}
    \item Модель способна сохранять пропорции приращения, заданные реальной мыщцой.
    \item Модель ведёт себя монотонно, то есть, например, при сокращении радиусы не перестают расти (или убывать, если коэффициент роста меньше 0), в то время как узел в реальном бицепсе при сокращении может как увеличиться, так и уменьшиться.
    \item Ввиду ограничения, связанного с постоянством объема, радиусы узлов модели получают прирост в 5-6 раз меньший, нежели узлы реальной мышцы.
\end{itemize}
