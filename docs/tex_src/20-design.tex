\chapter{Конструкторская часть}

В данном разделе предсатвлены требования к программному обеспечению, а также схемы алгоритмов, выбранных для решения поставленной задачи.

\section{Требования к программному обеспечению}

Программа должна предоставлять доступ к функционалу:
\begin{itemize}
    \item конфигурирование параметров мышцы через конфигурационный файл;
    \item конфигурирование параметров произвольного узла мышцы в интерактивном режиме;
    \item сокращение и растяжение модели;
    \item вращение, перемещение и масштабирование модели;
    \item изменение положения источника света.
\end{itemize}

К программе предъявляются следующие требования:
\begin{itemize}
    \item время отклика программы должно быть менее 1 секунды для корректной работы в интерактивном режиме;
    \item программа должна корректно реагировать на любые действия пользователя.
\end{itemize}

\clearpage
\section{Разработка алгоритмов}

Для алгоритмов, разработанных автором работы, представлены схемы алгоритмов. Для алгоритма синтеза изображения, использующего в своей основе алгоритмы Z-буфера\cite{zbuf} и Гуро\cite{lmodels} представлена блок-схема.

\subsection{Алгоритм деформации мышцы}

На рисунке \ref{img:deform} представлена схема алгоритма деформации мышцы.

\img{150mm}{deform}{Схема алгоритма деформации мышцы}

\clearpage
\subsection{Алгоритм триангуляции мышцы}


На рисунке \ref{img:triangulation} представлена схема алгоритма триангуляции мышцы.

\img{190mm}{triangulation}{Схема алгоритма триангуляции мышцы}

\clearpage
\subsection{Алгоритм синтеза изображения}

На рисунке \ref{img:graphics} представлена блок-схема алгоритма синтеза изображения.

\img{190mm}{graphics}{Блок-схема алгоритма синтеза изображения}

\section*{Вывод}

В данном разделе были представлены требования к программному обеспечению и разработаны схемы реализуемых алгоритмов.
