\usepackage{cmap} % Улучшенный поиск русских слов в полученном pdf-файле
\usepackage[T2A]{fontenc} % Поддержка русских букв
\usepackage[utf8]{inputenc} % Кодировка utf8
\usepackage[english,russian]{babel} % Языки: русский, английский
%\usepackage{pscyr} % Нормальные шрифты
\usepackage{pgfplots}% Пакет для создания графиков.

\usepackage{tabularx} % Пакет для создания таблиц с автоматически подбираемой шириной столбцов.
\pgfplotsset{compat=1.9} % Устанавливает версию совместимости для пакета pgfplots.
\usepackage{amsmath} % Пакет для расширенной поддержки математического типографирования.
\usepackage{geometry} % Пакет для настройки геометрии страницы.

% Задание отступов страницы:
\geometry{left=30mm}
\geometry{right=15mm}
\geometry{top=20mm}
\geometry{bottom=20mm}

% Пакет для настройки формата заголовков разделов и глав.
\usepackage{titlesec}
\titleformat{\section}
{\normalsize\bfseries}
{\thesection}
{1em}{}

% Задание интервалов между заголовками и текстом:
\titlespacing*{\chapter}{0pt}{-30pt}{8pt}
\titlespacing*{\section}{\parindent}{*0.5}{*0.5}
\titlespacing*{\subsection}{\parindent}{*0.5}{*0.5}
\titlespacing*{\section}{0pt}{12pt}{12pt}


% Пакет для установки интервалов между строками.
\usepackage{setspace}
\onehalfspacing % Полуторный интервал
\frenchspacing % Удаление дополнительного пробела после знаков препинания.
\usepackage{indentfirst} % Красная строка

% Дополнительное форматирование заголовков глав и разделов:
\usepackage{titlesec}
\titleformat{\chapter}{\LARGE\bfseries}{\thechapter}{20pt}{\LARGE\bfseries}
\titleformat{\section}{\Large\bfseries}{\thesection}{20pt}{\Large\bfseries}

\usepackage{listings}  	% Пакет для вставки блоков программного кода.
\usepackage{xcolor} 	% Пакет для использования цветов.
\usepackage{pdfpages}

\usepackage[normalem]{ulem} % \uline    % Пакет для подчеркивания текста и других эффектов
\renewcommand{\underset}[2]{\ensuremath{\mathop{\mbox{#2}}\limits_{\mbox{\scriptsize #1}}}} % Use \underset in normal environment, not only in math
\usepackage{xparse} % \NewDocumentCommand for creating custom commands  % Пакет для создания новых команд с более сложными аргументам


% (Здесь следуют определения новых команд, созданных с использованием xparse)
% Write stuff under underlined text
\NewDocumentCommand{\ulinetext}{O{3cm} O{c} m m} % O - optional; m - mandatory
{\underset{#3}{\uline{\makebox[#1][#2]{#4}}}}

% Test bold underline
\NewDocumentCommand{\bolduline}{}{\bgroup\markoverwith
	{\rule[-0.5ex]{2pt}{2pt}}\ULon}

% Display numberless sections in toc
\NewDocumentCommand{\snumberless}{m}
{\section*{#1}\addcontentsline{toc}{section}{#1}}



\usepackage{pgfplots}
\usetikzlibrary{datavisualization}
\usetikzlibrary{datavisualization.formats.functions}
 % Пакет для вставки графических изображений.
\usepackage{graphicx}


% Работа с изображениями и таблицами; переопределение названий по ГОСТу
\usepackage{caption}
\captionsetup[figure]{name={Рисунок},labelsep=endash}
\captionsetup[table]{singlelinecheck=false, labelsep=endash}

% (Здесь следуют новые команды для вставки изображений с различными параметрами)
\newcommand{\imgw}[3] {
	\begin{figure}[!ht]
		\center{\includegraphics[width=#1]{assets/img/#2}}
		\caption{#3}
		\label{img-#2}
	\end{figure}
}
\newcommand{\img}[3] {
	\begin{figure}[!ht]
		\center{\includegraphics[height=#1]{assets/img/#2}}
		\caption{#3}
		\label{img-#2}
	\end{figure}
}
\newcommand{\boximg}[3] {
	\begin{figure}[!ht]
		\center{\fbox{\includegraphics[height=#1]{assets/img/#2}}}
		\caption{#3}
		\label{img-#2}
	\end{figure}
}
\newcommand{\mycomment}[1]{}
\usepackage[justification=centering]{caption} % Настройка подписей float объектов
\usepackage[unicode,pdftex]{hyperref} % Ссылки в pdf
\hypersetup{hidelinks}
\newcommand{\code}[1]{\texttt{#1}}
\usepackage{icomma} % Интеллектуальные запятые для десятичных чисел
\usepackage{csvsimple}
\def\labelitemi{--} % Маркеры в ненумерованных списках
\usepackage{algorithm}
\usepackage{algpseudocode}
\usepackage{svg-extract}

\usepackage{caption}
\usepackage{threeparttable}
\usepackage{booktabs}