\chapter*{Заключение}
\addcontentsline{toc}{chapter}{Заключение}

В ходе данной практической работы была успешно достигнута поставленная цель --- освоены базовые принципы и алгоритмы, необходимые для создания программного обеспечения для визуализации и макетирования загородной местности.

Одним из важных этапов стал отбор подходящих структур данных, обеспечивающих эффективное хранение объектов сцены.  Ключевым моментом анализа подходов к моделированию трехмерных объектов стал выбор алгоритма Z-буфера для удаления невидимых ребер, а также его модификация для создания теней. Это обеспечило не только совместимость компонентов системы, но и упрощенную интеграцию.

Глубокий анализ алгоритмов и структур данных научил меня обдуманно и стратегически подходить к выбору конкретных методов решения задач. Этот процесс дал мне возможность лучше понимать, как каждый алгоритм может влиять на эффективность и функциональность разрабатываемого программного обеспечения, позволяя мне принимать обоснованные решения, направленные на оптимизацию и улучшение проекта.
