\chapter{Аналитический раздел}
% todo 
В данном разделе описаны способы описания моделей, 
\section{Способы определения моделей}

В системах трехмерного моделирования используются каркасные, поверхностные и объемные твердотельные модели. 
Правильный выбор метода определения моделей на сцене определяет размер и визуализацию модели на сцене.

\textbf{Каркасная модель} --- это простейшая модель трехмерного объекта, представляющая собой совокупность вершин, соединенных между собой ребрами~\cite{model_geom}.

Главным недостатком данной модели является отсутствие информации о поверхности объекта, что делает невозможным разграничение внутренних и внешних граней, например, как на рисунке~\ref{img:mod1}.
%Этот недостаток заметен, когда у объекта есть отверстия, например, как на рисунке~\ref{img:mod1}, где неясно, какие грани связаны с отверстием.
Каркасная модель занимает меньше памяти и эффективна для простых задач~\cite{model_geom_01}.
%Широко применяется для имитации траектории движения инструмента, выполняющего простые операции~\cite{model_geom_01}.
 

\includeimage
	{mod1} % Имя файла без расширения (файл должен быть расположен в директории inc/img/)
	{f} % Обтекание (без обтекания)
	{h} % Положение рисунка (см. figure из пакета float)
	{0.3\textwidth} % Ширина рисунка
	{Пример каркасной модели} % Подпись рисунк
	

\textbf{Поверхностная модель}, в отличие от каркасной, включает в себя не только вершины и ребра, но также поверхности, создавая визуальный контур объекта~\cite{model_geom}. Каждый объект в данной модели обладает внутренней и внешней частью, как на  рисуноке~\ref{img:mod2}.

В основу поверхностной модели положены два основных математических положения:
\begin{itemize}
	\item любую поверхность можно аппроксимировать многогранником, где каждая грань представляет собой простейший плоский многоугольник~\cite{model_geom_01};
	\item в модели допускаются не только плоские многоугольники, но и поверхности второго порядка, а также аналитически неописываемые поверхности~\cite{model_geom_01}.
\end{itemize}

\includeimage
	{mod2} % Имя файла без расширения (файл должен быть расположен в директории inc/img/)
	{f} % Обтекание (без обтекания)
	{h} % Положение рисунка (см. figure из пакета float)
	{0.3\textwidth} % Ширина рисунка
	{Пример поверхностной модели} % Подпись рисунк

%Поверхностные модели широко применяются в разработке динамических поверхностей, которые взаимодействуют с внешней средой. Примеры таких поверхностей включают фюзеляжи самолетов, обводы судов, кузова автомобилей

Недостаток поверхностной модели --- отсутствует информация, о том, с какой стороны поверхности находится материал, а с какой пустота.

\textbf{Твердотельная модель} отличается от поверхностной тем, что включает информацию о расположение материала с обеих сторон поверхности~\cite{model_geom_01}.

\textbf{Выбор определения моделей} 

В данной задаче наиболее оптимальными являются поверхностные модели объектов, так как каркасные модели недостаточно полно представляют форму объекта, а твердотельные модели избыточны.


\section{Методы представления трехмерных поверхностей}

\textbf{Метод полигональной сетки} представляет объект в виде связанной между собой сетки плоских многоугольников~\cite{model_geom}, как показано на рисунке~\ref{img:mod3}.

\textbf{Метод параметрических бикубических кусков} использует математические формулы, описывающие координаты поверхностей.
Этот подход обеспечивает высокую точность при описании поверхности и требует меньше элементов для представления сложных форм, в сравнении с полигональными сетками.
Однако алгоритмы, работающие с бикубическими кусками сложнее~\cite{model_geom}.

В рамках данной задачи выбран метод полигональной сетки. Этот выбор обоснован геометрической простотой объектов сцены и отсутствием необходимости использования сложных математических формул.
Использование полигональной сетки позволит применять более простые алгоритмы для обработки объектов~\cite{model_geom}.

\includeimage
	{mod3} % Имя файла без расширения (файл должен быть расположен в директории inc/img/)
	{f} % Обтекание (без обтекания)
	{h} % Положение рисунка (см. figure из пакета float)
	{0.3\textwidth} % Ширина рисунка
	{Пример полигональной сетки, изображающей кролика} % Подпись рисунк


\textbf{Способы описания полигональных сеток}

Наиболее распространенные методы представления полигональных сеток, рассматриваются в \cite{polygon_mesh}.

%todo
\begin{enumerate}
	\item Список граней ---  распространенный метод представления трехмерных моделей, описывает объект как множество граней и вершин, где каждая грань имеет минимум 3 вершины.
	\textit{Преимущества:} простота поиска вершин грани, динамическое обновление формы без изменения связности граней.
	\textit{Недостатки:} трудности при операциях разрыва и объединения граней, а также проблемы с поиском граней.
	
	\item Вершинное представление --- это метод представления модели через множества вершин, которые связаны между собой.
	В качестве \textit{преимущества} можно выделить его простату.
	К \textit{недостаткам:} отсутствие явной информации о гранях и ребрах, а также редкое использование в современных системах визуализации.
	
	\item <<Крылатое>> представление --- метод, представляющий модель как упорядоченное множество граней вокруг ребра.
	\textit{Преимущество:} решение проблемы перехода от ребра к ребру через упорядоченное множество граней.
	\textit{Недостаток:} высокие требования к памяти из-за увеличивающейся сложности структуры
\end{enumerate}


\textbf{Выбор способа описания полигональной сетки} % мб не надо 

Для работы выбран метод представления моделей через список граней, обеспечивающий ясное описание и удобный доступ к элементам сетки. Этот подход упрощает модификацию моделей, включая добавление, удаление и изменение граней и вершин.
 %Благодаря этому методу, я могу выполнять различные операции, такие как вычисление нормалей.

% Важно учесть, что, несмотря на высокую степень гибкости и удобства работы, использование списка граней может потребовать больше памяти по сравнению с другими методами. Однако, учитывая доступные ресурсы памяти и объем моделей, с которыми я работаю, преимущества этого подхода перевешивают его потенциальные недостатки.




\section*{Вывод}

Анализ подходов к заданию трехмерных моделей привел к выбору поверхностных моделей, метода полигональных сеток в качестве способа представления трехмерных поверхностей. Для удаления невидимых ребер применен алгоритм Z-буфера. Модификация Z-буфера использована для построения теней, обеспечивая совместимость и упрощенную интеграцию. 