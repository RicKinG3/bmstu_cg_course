\chapter{Технологический раздел}

В данном разделе будут приведены средства реализации разработки программного обеспечения, сведения о модулях программы, листинги кода реализации алгоритмов и интерфейс программного обеспечения.

\section{Средства реализации}

В качестве языка программирования для разработки программного обеспечения был выбран язык \textit{С++}~\cite{cplusplus}.
Данный выбор обусловлен следующим: 
\begin{enumerate}
	\item возможность реализовать все алгоритмы, выбранные в результате проектирования.
	\item поддержка языком всех типов и структур данных, выбранных в результате
	проектирования;
\end{enumerate}

\section{Сведения о модулях программы}
Программа состоит из трех программных модулей:
\begin{enumerate}
	\item \texttt{main.cpp} --- файл, который включает в себя точку входа программы.
	\item \texttt{mainwindow.cpp, mainwindow.h} --- модуль пользовательского интерфейса;
	\item \texttt{mainwindow.ui} --- форма пользовательского интерфейса;
	\item \texttt{Drawer.cpp, Drawer.h} --- модуль вывода результата работы программы на экран;
	\item \texttt{Facade.cpp, Facade.h} --- модуль реализующий действия пользователя;
	\item \texttt{Platform.cpp, Platform.h} --- модуль реализующий действия с сценой (платформой на которую устанавливаются объекты);
	\item \texttt{Model.cpp, Model.h} --- модуль реализующий действия с объектами;
	\item \texttt{Point.cpp, Point.h} --- модуль реализующий действия с точками объекта;
	\item \texttt{Polygon.cpp, Polygon.h} --- модуль реализующий действия с полигонами объекта;
	\item \texttt{Vertex.cpp, Vertex.h} --- модуль реализующий действия с вершинами объекта;
	\item \texttt{Light.cpp, Light.h} --- модуль реализующий действия с источником света;
\end{enumerate}


%todo
\section{Структура реализуемых классов}

На рисунке~\ref{img:class} представлена UML схема для реализуемого ПО

\includeimage
{class} % Имя файла без расширения (файл должен быть расположен в директории inc/img/)
{f} % Обтекание (без обтекания)
{h} % Положение рисунка (см. figure из пакета float)
{1\textwidth} % Ширина рисунка
{схема UML} % Подпись рисунк



\section{Реализация алгоритмов}

В листингe~\ref{lst:z-alg.cpp} приведен фрагмент реализации модифицированной версии алгоритма Z-буфера для построения теней.
	
\newpage
\includelistingpretty
{z-alg.cpp} % Имя файла с расширением (файл должен быть расположен в директории inc/lst/)
{c++} % Язык программирования (необязательный аргумент)
{фрагмент реализации модифицированного алгоритма Z-буфера} % Подпись листинга

