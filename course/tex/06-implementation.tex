\chapter{Технологический раздел}

В данном разделе будут приведены средства реализации разработки программного обеспечения, сведения о модулях программы, листинги кода реализации алгоритмов и интерфейс программного обеспечения.

\section{Средства реализации}

В качестве языка программирования для разработки программного обеспечения был выбран язык \textit{С++}~\cite{cplusplus}.
Данный выбор обусловлен следующим: 
\begin{enumerate}
	\item возможность реализовать все алгоритмы, выбранные в результате проектирования.
	\item поддержка языком всех типов и структур данных, выбранных в результате
	проектирования;
\end{enumerate}


\section{Сведения о модулях программы}
Программа состоит из следующих программных модулей и файлов:
\begin{enumerate}
	\item \texttt{main.cpp} --- файл, который содержит точку начала выполнения программы.
	\item \texttt{mainwindow.cpp, mainwindow.h} --- модуль пользовательского интерфейса, включает описание и реализацию класса \textit{MainWindow};
	\item \texttt{mainwindow.ui} --- это файл, который описывает пользовательский интерфейс программы;
	\item \texttt{Drawer.cpp, Drawer.h} --- модуль, который содержит реализацию модифицированного алгоритма Z-буфера и включает описание и реализацию класса \textit{Drawer}, который отвечает за отображение результатов работы программы на экране;
	\item \texttt{Facade.cpp, Facade.h} --- модуль реализующий действия пользователя, включает описание и реализацию класса \textit{Facade};
	\item \texttt{Platform.cpp, Platform.h} --- модуль, который содержит реализацию действий с платформой, на которую устанавливаются объекты, а также описывает и реализует класс \textit{Platform}
	\item \texttt{Model.cpp, Model.h} --- модуль реализующий действия с объектами, включает описание и реализацию класса \textit{Model};
	\item \texttt{Point.cpp, Point.h} --- модуль реализующий действия с точками объекта, включает описание и реализацию класса \textit{Point};
	\item \texttt{Polygon.cpp, Polygon.h} --- модуль реализующий действия с полигонами объекта, включает описание и реализацию класса \textit{Polygon};
	\item \texttt{Vertex.cpp, Vertex.h} --- модуль реализующий действия с вершинами объекта, включает описание и реализацию класса \textit{Vertex};
	\item \texttt{Light.cpp, Light.h} --- модуль реализующий действия с источниками света, включает описание и реализацию класса \textit{Light};
\end{enumerate}


%todo
\section{Диаграмма классов}

На рисунке~\ref{img:class} представлена UML-диаграмма классов разработанного ПО
\clearpage
\includeimage
{class} % Имя файла без расширения (файл должен быть расположен в директории inc/img/)
{f} % Обтекание (без обтекания)
{h} % Положение рисунка (см. figure из пакета float)
{1\textwidth} % Ширина рисунка
{UML-диаграмма классов} % Подпись рисунк
\clearpage


\section{Реализация алгоритмов}

%todo   норм ли переносы и норм ли  сокращения 
В листингe~\ref{lst:z-alg.cpp} приведен фрагмент реализации модифицированной версии алгоритма Z-буфера для построения теней.
	
\includelistingpretty
{z-alg.cpp} % Имя файла с расширением (файл должен быть расположен в директории inc/lst/)
{c++} % Язык программирования (необязательный аргумент)
{фрагмент реализации модифицированного алгоритма Z-буфера} % Подпись листинга

%todo
\section{Интерфейс программного обеспечения}

На рисунке~\ref{img:interface-obj} представлен интерфейс программного обеспечения, конкретно вкладки "Объекты". Этот интерфейс позволяет:
\begin{itemize}
	\item выбирать объекты, которые можно установить на платформу;
	\item выбрать расположение объектов на платформе;
	\item установить выбранные объекты на платформу с выбранным;
	\item переместить или удалить выбранный объект;
	\item добавить источник света с выбранными углами по вертикали и горизонтали.
\end{itemize}

\includeimage
	{interface-obj} % Имя файла без расширения (файл должен быть расположен в директории inc/img/)
	{f} % Обтекание (без обтекания)
	{h} % Положение рисунка (см. figure из пакета float)
	{0.4\textwidth} % Ширина рисунка
	{Интерфейс программного обеспечения, вкладка "Объекты"} % Подпись рисунк


%todo
На рисунке~\ref{img:interface-platform} представлен интерфейс программного обеспечения, конкретно вкладки "Платформа". Этот интерфейс позволяет:
\begin{itemize}
	\item выбирать размер платформы;
	\item вращать платформу с объектами;
	\item масштабировать платформу с объектами;
	\item перемещать платформу с объектами;
	\item расширять платформу.
\end{itemize}

\includeimage
{interface-platform} % Имя файла без расширения (файл должен быть расположен в директории inc/img/)
{f} % Обтекание (без обтекания)
{h} % Положение рисунка (см. figure из пакета float)
{0.4\textwidth} % Ширина рисунка
{Интерфейс программного обеспечения, вкладка "Платформа"} % Подпись рисунк



\section*{Вывод}

В этом было разработано программное обеспечение, были приведены средства реализации разработки программного обеспечения, сведения о модулях программы, листинги кода реализации алгоритмов и интерфейс программного обеспечения.
